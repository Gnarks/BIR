\documentclass[a4paper, 12pt]{article}
\usepackage[english]{babel}
\usepackage[pdftex]{graphicx}
\usepackage[T1]{fontenc}
\usepackage{wrapfig}
\usepackage{graphicx} 
\usepackage{algorithm}
\usepackage{hyperref}
\usepackage{amssymb}

\newcommand{\HRule}{\rule{\linewidth}{0.3mm}}

\begin{document}
\begin{center}
% Title
\HRule \\[0.3cm]
{ \LARGE \bfseries Lab visit report \\[0.3cm]}
{ \LARGE \bfseries 03/07/25 \\[0.1cm]} % Commenter si pas besoin
\HRule \\[1.5cm]

% Author and supervisor
\begin{minipage}[t]{0.45\textwidth}
\begin{flushleft} \large
\textsc{supervisors :}\\
Bruno \textsc{Quoitin}\\
Aqeel \textsc{Ahmed}\\
\end{flushleft}

\end{minipage}
\begin{minipage}[t]{0.45\textwidth}
\begin{flushright} \large
\textsc{author :}\\
Maxime \textsc{Bartha}\\
\end{flushright}
\end{minipage}\\[2ex]
\end{center}


\section{Initial objectives}
\begin{itemize}
  \item implement Tx side proof of concept 
  \item get a simple capture
  \item check for the sleeping drift 
\end{itemize}

\section{Material and software used}

\section{Summary}
TCP connection (local)  between Rx and Tx done
sleeping drift might be a bigger issue the anticipated
slotted transmissions finished 

\section{Problems encountered} 
even with the embedded python file, the init waiting for the TCP connection causes grc to freeze

\section{Solutions found}
send packet for the init state to be finished. Then it is possible to work on grc again (each time the python file is savec, the init function is called freezing grc)

\section{Conclusion}
tommorow : connect the transmitter and receveir by TCP and try the file saver

improvement : add Quoitin's code to the arduino code + clean it
add frequency and number of cycles as parameters for arduino (need to transfer more than 1 byte)

\end{document}
