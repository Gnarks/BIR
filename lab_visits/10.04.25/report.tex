\documentclass[a4paper, 12pt]{article}
\usepackage[english]{babel}
\usepackage[pdftex]{graphicx}
\usepackage[T1]{fontenc}
\usepackage{wrapfig}
\usepackage{graphicx} 
\usepackage{algorithm}
\usepackage{hyperref}
\usepackage{amssymb}

\newcommand{\HRule}{\rule{\linewidth}{0.3mm}}

\begin{document}
\begin{center}
% Title
\HRule \\[0.3cm]
{ \LARGE \bfseries Lab visit report \\[0.3cm]}
{ \LARGE \bfseries XX/XX/25 \\[0.1cm]} % Commenter si pas besoin
\HRule \\[1.5cm]

% Author and supervisor
\begin{minipage}[t]{0.45\textwidth}
\begin{flushleft} \large
\textsc{supervisors :}\\
Bruno \textsc{Quoitin}\\
Aqeel \textsc{Ahmed}\\
\end{flushleft}

\end{minipage}
\begin{minipage}[t]{0.45\textwidth}
\begin{flushright} \large
\textsc{author :}\\
Maxime \textsc{Bartha}\\
\end{flushright}
\end{minipage}\\[2ex]
\end{center}


\section{Initial objectives}
Get point-to-point communication working between 2 MKR1310 boards.\\
Try to get some signals through SDRs and plot them in URH.

\section{Material and software used}

\begin{itemize}
  \item Multiple Arduino MKR1310
  \item Multiple usb hubs
  \item Usb docking station
  \item Oscilloscope
  \item SDR device
  \item Continuous current generator
  \item Arduino IDE
  \item URH
\end{itemize}

electrical signal reader
continuous current electrical generator 

\section{Summary}
After testing the board and the antenna separately, I figured one of the boards was indeed defective.\\ Trying another board worked. \\
Point-to-point communication didn't work out of the box with Arduino IDE examples.\\ Adding : \\
\emph{
LoRa.setSignalBandwidth(125E3); \\
LoRa.setSpreadingFactor(8);  \\
LoRa.setCodingRate4(5);}\\
didn't solve the issue.

Mr.Quoitin and Mr.Aqeel came to the lab, we figured that the power supply was the issue.\\
We looked at the voltage with an oscilloscope doing through the boards and found and undersupply of power. 

After figuring the power issues, we tried to connect an SDR device to my laptop and listen for signals.\\
Without success because of some missing libraries/drivers ?
The SDR device, even if registered by my laptop as connected, wouldn't show on URH.


\section{Problems encountered} 
Under power issue with the boards.\\
SDR device not recognized by URH.

\section{Solutions found}
For the under power issue :\\
Either use the (better) usb hub from Taiwan and supply it with a current generator.
Or power all the boards are by connecting them directly to a current generator (+ to Vin pin and - to grnd).

\section{Conclusion}
We could figure out a good way of powering the boards.


\section{Next time}
I'll get the SDR device working to finally listen to (Multiple ?) LoRa transmissions.
I can take a closer look at the power issues.



\end{document}
