\documentclass[a4paper, 12pt]{article}
\usepackage[english]{babel}
\usepackage[pdftex]{graphicx}
\usepackage[T1]{fontenc}
\usepackage{wrapfig}
\usepackage{graphicx} 
\usepackage{algorithm}
\usepackage{hyperref}
\usepackage{amssymb}

\newcommand{\HRule}{\rule{\linewidth}{0.3mm}}

\begin{document}
\begin{center}
% Title
\HRule \\[0.3cm]
{ \LARGE \bfseries Lab visit report \\[0.3cm]}
{ \LARGE \bfseries 30/06/25 \\[0.1cm]} % Commenter si pas besoin
\HRule \\[1.5cm]

% Author and supervisor
\begin{minipage}[t]{0.45\textwidth}
\begin{flushleft} \large
\textsc{supervisors :}\\
Bruno \textsc{Quoitin}\\
Aqeel \textsc{Ahmed}\\
\end{flushleft}

\end{minipage}
\begin{minipage}[t]{0.45\textwidth}
\begin{flushright} \large
\textsc{author :}\\
Maxime \textsc{Bartha}\\
\end{flushright}
\end{minipage}\\[2ex]
\end{center}


\section{Initial objectives}
\begin{enumerate}
  \item get my computer setup for all sdrs with gnu radio
  \item plan the rest of the internship
  \item make a python script to estimate the time for specific scenario (SF, \#Dev, \#frames,..)
  \item lora communication with 2 MKR1310
  \item plot the lora communication to see the Chirps 
  \item read Lora Gnu SDR implementation 
\end{enumerate}

\section{Material and software used}
\begin{itemize}
  \item arduino MKR1310
  \item USRP SDR
  \item pluto SDR
  \item cable communication
\end{itemize}

\section{Summary}
morning : 1, 2 points done 
afternoon : 3,4, (5,6)

I installed all the necessary dependencies for every SDR.

Made the python script 
Had time to comm between 2 MKR1310



\section{Problems encountered} 
cmake dependencies

\section{Solutions found}
check for cmake dep

\section{Conclusion}
Tommorow : test with 1 sdr and 1 transmitter the minimum interval needed between 2 frames for the receiver to detect them correctly
to get an estimation of the time between 2 frames needed

take a look at Quoitin's code 


memo : settled on sending frames each transmitter at a time and cycle until all frames per device are sent
\end{document}
